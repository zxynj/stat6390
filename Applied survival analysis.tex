\documentclass[a4paper,11pt]{article}

\usepackage[latin1]{inputenc}
\usepackage[T1]{fontenc}
\usepackage{graphicx}
\usepackage{booktabs}
\usepackage{subfigure}
\usepackage{enumerate}
\usepackage{amsmath}
\usepackage{hyperref}
\usepackage{fancyvrb}

\fvset{frame=single,framesep=1mm,fontfamily=courier,fontsize=\scriptsize,
	numbers=left,framerule=.3mm,numbersep=1mm,commandchars=\\\{\}}

\title{Applied survival analysis notes}
\author{}
\date{}

\setlength{\textheight}{11in}
\setlength{\textwidth}{6.9in}
\setlength{\topmargin}{-.2in}
\setlength{\oddsidemargin}{-.2in}
\setlength{\evensidemargin}{-.2in}
\setlength{\headsep}{0in}

\newcommand{\bigbrk}{\vspace*{2in}}
\newcommand{\smallbrk}{\vspace*{.3in}}

\pagenumbering{gobble} % turn off page #

\begin{document}
\maketitle

\paragraph{Chapter 2: Descriptive methods for survival data}
\begin{itemize}
\item In the example, no two subjects shared an observation time, and the longest observed time was a failure. 
Simple modifications to the method described above are required when either of these conditions is not met.
\end{itemize}
\end{document}
