\documentclass[10pt]{article}
\usepackage[margin=0.5in]{geometry}
\usepackage{xcolor}
\usepackage{graphicx}
\usepackage{fancyhdr}
\usepackage{tgschola}
\usepackage{lastpage}
\usepackage{amsmath}
\usepackage{booktabs}
\usepackage{multirow}
\usepackage{enumerate}
\usepackage{ hyperref}
 
\addtolength{\oddsidemargin}{0in}
\addtolength{\evensidemargin}{0in}
\addtolength{\textwidth}{0in}

\def\red{\textcolor[rgb]{1,0,0}}
\def\R{\texttt R}

\begin{document}

\begin{center}
\renewcommand{\arraystretch}{1.2}
\begin{tabular}{|p{.2\textwidth}| p{.3\textwidth} | p{.6\textwidth}|}
\hline
\vspace{.03cm}
\multirow{ 2}{*}{\includegraphics[scale = .2, trim = 0cm 3cm 3cm 9.5cm, clip]{utdtwo}} & 
\multicolumn{1}{r}{\textbf{Course}} & Probability and Statistics for Data Science and Bioinformatics\\
& \multicolumn{1}{r}{\textbf{Class number}} & Stat 6390.001 \\
& \multicolumn{1}{r}{\textbf{Professor}} & Sy Han (Steven) Chiou\\
& \multicolumn{1}{r}{\textbf{Term}} & Fall 2018\\
& \multicolumn{1}{r}{\textbf{Schedule}} & Tuesday, Thursday, 10:00 am-11:15 am, CB 1.214\\
\hline
\end{tabular}
\end{center}

\paragraph{Professor's Contact Information}
\begin{center}
\renewcommand{\arraystretch}{1.2}
\begin{tabular}{|p{.2\textwidth} p{.75\textwidth}|}
\hline
\textbf{Office Phone} & 972.883.6362 \\
\textbf{Office Location} & FO 2.410A\\
\textbf{Email address} & schiou@utdallas.edu\\
\textbf{Course website} & \url{http://elearning.utdallas.edu/} \newline
All course related materials, including lecture notes, will be posted here.\\
\textbf{Office Hours} & Tuesday, Thursday, 12:30 pm - 1:30 pm or by appointment.\\
%\textbf{Teaching Assistant} & ?? \\
\hline
\end{tabular}
\end{center}

\paragraph{General Course Information}
\begin{center}
\renewcommand{\arraystretch}{1.2}
\begin{tabular}{|p{.2\textwidth} p{.75\textwidth}|}
\hline
\textbf{Prerequisite} & Calculus through multivariate calculus and basic knowledge of regression methods, statistical methods for estimation and inferences. \\
[.5ex]
\textbf{Course Coverage} & The course will cover the basic concepts and methods for analyzing survival time data.
Some of the key topics to be covered are: 
characteristics of survival data, 
Nelson-Aalen estimator, Kaplan-Meier estimator, log-rank tests,
counting processes and martingales, 
estimation and inference methods for parametric survival models, proportional hazard models, accelerated failure time models.
Statistical software, R, will be introduced in the lectures and used for homework assignments and projects.\\
[.5ex]
\textbf{Learning outcomes} & 
As a result of completing this course, students should be able to identify survival data and apply appropriate methods in practical settings.
For students who are looking for a research topic in survival analysis, a thorough understanding of the presented theories and additional readings will be beneficial.
\\
[.5ex]
\textbf{Required Text} & \textit{Applied survival analysis: Regression modeling of time-to-event data}, second edition by David W. Hosmer, Stanley Lemeshow, and Susanne May.
ISBN: 978-0-471-75499-2\\
\hline
\end{tabular}
\end{center}


\paragraph{Course Policies}
\begin{center}
\renewcommand{\arraystretch}{1.2}
\begin{tabular}{|p{.2\textwidth} p{.75\textwidth}|}
\hline
\textbf{Course grade} & \textbf{Homework: 50\%}: \newline
Homework will be assigned on a roughly biweekly basis.
The assignments should be turned in \emph{in person} and be prepared using either \emph{R markdown} or \emph{knitr}.
\newline
\textbf{Exams (1 midterm \& 1 Final): 25\%}
\newline
These exams will contain a take-home portion that requires \texttt{R} and in-class a portion that are closed-book and closed-notes. 
No make-up exams are allowed unless a special arrangement made \textbf{\emph{in advance}}. 
Missed exam due to oversleeping, car troubles, forgetfulness, etc., are not excused.
The final exam date and time will be announced when it is available. \\
[.8ex]\textbf{Letter grades} & 
\textbf{A+}: 96 - 100 \hspace{.2cm} \textbf{A}: 93 -- 95.99 \hspace{.2cm} \textbf{A--}: 90 -- 92.99 \\
&\textbf{B+}: 86 - 90 \hspace{.4cm} \textbf{B}: 83 -- 85.99 \hspace{.2cm} \textbf{B--}: 80 -- 82.99 \\
& \textbf{C+}: 76 - 80 \hspace{.4cm} \textbf{C}: 73 -- 75.99 \hspace{.2cm} \textbf{C--}: 60 -- 72.99 \hspace{.2cm} \textbf{F}: 0 -- 59.99.\\
[.8ex]\textbf{Policy on the use of electronic devices} & 
For many students, using laptops or other personal computing devices in lecture is an
efficient way to read lecture slides and take notes. However, using these in ways that
are not related to course work can be distracting to other nearby students.
Please limit the use of personal computing devices in lecture to activities
directly related to the lecture. \\
[.8ex]\textbf{Academic integrity} & The faculty expects from students a high level of responsibility and academic honesty. Scholastic dishonesty includes, but is not limited to, cheating, plagiarism, collusion, and falsifying of records. Violators face disciplinary proceedings.\\
[.8ex]\textbf{Withdrawal} & Deadlines for withdrawal from courses are published in each semester's course catalog. A faculty member cannot drop or withdraw a student. It is the student's responsibility to handle withdrawal procedures from any class to avoid receiving a grade of ``F''. \\
%\textbf{UT Dallas syllabus policies and procedures} & The information contained in the following link constitutes the University's policies and procedures segment of the course syllabus. Please go to \url{http://go.utdallas.edu/syllabus-policies} for these policies.\\
\hline
\end{tabular}
\end{center}



\newpage
\paragraph{Tentative Course Schedule (subject to change)}
\begin{center}
\renewcommand{\arraystretch}{1.3}
\begin{tabular}{|p{.15\textwidth} p{.09\textwidth} p{.69\textwidth} |}
\hline
& \textbf{Coverage} & \textbf{Topics} \\
\textbf{Week 1 (8/21)} & Chapter 1 & Typical censoring and truncation mechanisms\\
\textbf{Week 2 (8/28)} & Chapter 2 & Estimating the survival function (Kaplan-Meier estimator)\\
\textbf{Week 3 (9/4)} & Chapter 2 & Comparison of survival functions (log-rank test)\\
\textbf{Week 4 (9/11)} & Chapter 2 & Other functions of survival time and their estimators (Nelson-Aalen estimator)\\
\textbf{Week 5 (9/18)} & Chapter 2 & Other functions of survival time and their estimators (Nelson-Aalen estimator)\\
\textbf{Week 6 (9/25)} & Chapter 8 & Parametric regression models\\
\textbf{Week 7 (10/2)} & Chapter 8 & Parametric regression models\\
\textbf{Week 8 (10/9)} & Exam & \\
\textbf{Week 9 (10/16)} & Chapter 3 & Proportional hazards regression model\\
\textbf{Week 10 (10/23)} & Chapter 3 & Proportional hazards regression model\\
\textbf{Week 11 (10/30)} & Chapter 4 & Interpretation of a fitted proportional hazards regression model\\
\textbf{Week 12 (11/6)} & Chapter 4 & Interpretation of a fitted proportional hazards regression model\\
\textbf{Week 13 (11/13)} & Chapter 5 & Model development\\
\textbf{Week 14 (11/27)} & Chapter 5 & Model development\\
\textbf{Week 15 (11/4)} & Chapter 9 & Other models (accelerated failure time model and others)\\
\hline
\end{tabular}
\end{center}

\paragraph{More  Policies}
\begin{center}
\begin{tabular}{|p{.2\textwidth} p{.75\textwidth}|}
\hline
\textbf{Incomplete grades} & As per university policy, incomplete grades are granted only in the case of work unavoidably missed (and excused) and not already covered by the professor's policy on missed work or activities, and only if at least 70\% of the course work has been completed. An incomplete grade must be resolved within eight weeks from the first day of the subsequent long semester. If the required work to complete the course and to remove the incomplete grade is not submitted by the specified deadline, the incomplete grade becomes changed automatically to F. \\
[.8ex]\textbf{Student conduct and discipline} & The University of Texas System and The University of Texas at Dallas have rules and regulations for the orderly and efficient conduct of university business.  See the UTD publication, A to Z Guide, issued to each registered student.\\
[.8ex]\textbf{Disability services} & Disability Services seeks to provide students with disabilities educational opportunities equivalent to those of their non-disabled peers. The Office of Disability Services is located in room 1.610 in the Student Union, and its hours are Monday-Thursday 8:30 a.m. to 6:30 p.m. and Friday 8:30 a.m. to 5:00 p.m. Essentially, the law requires colleges and universities to make reasonable adjustments necessary to eliminate discrimination on the basis of disability. For example, it may be necessary to remove classroom prohibitions against tape recorders or animals (in the case of dog guides) for students who are blind. Occasionally, an assignment requirement may be modified (for example, a research paper versus an oral presentation for a student who is hearing impaired).Classes including students with mobility impairments may have to be rescheduled in accessible facilities. The college or university may need to provide special services such as registration, note-taking, or mobility assistance. The student should notify the professor of the need for such accommodations. Disability Services provides students with letters to present to faculty members.\\
[.8ex] \textbf{Syllabus policies} & he information contained in the following link constitutes the University's policies and procedures segment of the course syllabus. Please go to \url{http://go.utdallas.edu/syllabus-policies} for these policies.\\
%[.8ex]\textbf{Religious holy days} & The University of Texas at Dallas excuses students from class or other required activities for the purpose of travel to and observance of a religious holy day for a religion whose places of worship are exempt from property tax under Section 11.20, Tax Code, Texas Code Annotated. In the case of such an absence, the student is encouraged to notify the instructor as soon as possible, preferably in advance. Missed assignments, quizzes, tests, or exams, will be covered by the professor's policy for excused missed or late work.\\
%[.8ex]\textbf{Copyright notice} & A UTD student is required to follow the UTD copyright policy. See \url{http://www.utsystem.edu/ogc/intellectualproperty/copypol2.htm}.\\
%\textbf{UT Dallas syllabus policies and procedures} & The information contained in the following link constitutes the University's policies and procedures segment of the course syllabus. Please go to \url{http://go.utdallas.edu/syllabus-policies} for these policies.\\
\hline
\end{tabular}
\end{center}

\end{document}


